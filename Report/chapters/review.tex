The main paper that we will form a basis of our research is Adamtzky's "Towards a Fungal Computer". \cite{fungalcomp} In the paper, Adamtzky claims that fungi of the genus Basidiomycetes are phenomenologically similar to the slime mold physarum polycephalum. This includes not only the network topology of fungal mycelium, but also its path-finding properties. The author identifies pleurotus ostreatus, i.e. the Oyster mushroom, as a potential candidate for conducting future network science and computing experiments. However, there is currently a gap in the literature of oyster mushroom mycelium and its path-finding properties; the extent to which it is similar to slime mold, and what are the differences, if any. Our research therefore aims to bridge the gaps in this shortage of knowledge by performing experiments that establish the similarity of mycelium to the slime mold.

Another relevant paper is "Rules for Biologically Inspired Adaptive Network Design" \cite{rulesBioDesign} by Tero and others. This paper serves as a foundation for the kind of network science experiments done using physarum polycephalum as a networking agent. We seek to replicate the experiments of this research, but using oyster mushroom mycelium and mapping Karachi's transport network instead of physarum mapping Tokyo's transport network (i.e. subway). The experiment here seeks to replicate a possible map of a subway/underground metro project in Karachi with different terminals, but using a novel agent (pleurotus ostreatus). The end-result of this publication, i.e. producing Minimum Spanning Trees (MSTs) of the optimized network provides a possible outcome for our final experimental results.

The publication "Maze-solving by an
amoeboid organism" \cite{mazesolving} establishes an experimental procedure for maze-solving experiments using Physarum. We will adapt some experimental procedures from this paper to our experiments using oyster mushroom mycelium. Particularly useful to us here is the laboratory setup for creating terrain for the biological agent to colonize.

"Network Organisation of Mycelial Fungi" \cite{networkmyc} is another useful publication which provides an extensive framework for the network representation of physical mycelium as graphs. This includes the adjacency-matrix representation of mycelium as a graph with vertices and edges, comparison of the proposed model with existing solutions (such as the cellular automaton representation of mycelium), as well as some network topology measures (i.e. betweenness centrality) and how they can be applied to mycelial networks. Particularly useful here are some image processing methods for extracting network data from mycelial networks that the authors use in their experimental analysis of live laboratory cultures.

The publication "Physarum can compute shortest paths" \cite{phy_short_path} provides a theoretical framework for mathematically proving the shortest-path finding properties of the slime mold. The authors use an axiomatic approach, modeling flux through the slime mold's capillary tubes and deriving subsequent results from the mathematical model. We intend on using this publication as a basis for deriving a similar result for the path-finding properties of oyster mushroom mycelium.

"Solving the Steiner Tree Problem in Graphs
using Physarum-inspired Algorithms" \cite{steiner_tree} also establishes a heuristic approach to design algorithms for the slime mold's ability to solve the Steiner tree proble; a super-set of the class of problems that include the shortest path problems. The author designs an algorithm using experimental data, building on earlier work in the field and tying experimental results to an actual algorithm. Our research aims to build a similar algorithm, but for Osyter mushroom mycelium rather than the slime mold.

Finally, another publication by Sun, "Physarum-inspired Network Optimization: A
Review" \cite{phy_network_review} provides an overview of the computing and network science experiments using slime molds so far. The experimental approach the author outlines to establish theoretical results from bio-agent data is what we will adapt, but to the oyster mushroom mycelium rather than the slime mold.

We will keep updating this chapter (especially if our project is research-intensive) as our research proceeds and we come across more work related to our problem.